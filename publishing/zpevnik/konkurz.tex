\section{Konkurz}

\thispagestyle{empty}

\begin{multicols}{2}


\begin{verse}

Napadlo mě tak nakonec roku,\\
že bych se mohl věnovat rocku.\\
Rád bych si zahrál k poslechu, tanci\\
gotiku, třeba, či renesanci.

Mrknul jsem kolem, kdo to tak hrává,\\
která že grupa bude ta pravá.\\
Nesmí to býti žabaři žádní,\\
nejlepší zdá se ten bigbít hradní.

Šel jsem na konkurz, ztratil den celý,\\
kapelník hlavu (jak) tuřanské zelí.\\
Vytáh’ jsem nástroj, prohrábl struny,\\
a on mi dává pokynů tuny.

Kdo chce s námi zpívat, kdo chce s námi hrát\\
ten se musí dívat na můj prstoklad\\
invenci si nežádám, vše vím nejlíp sám\\
a komu se to nezdá, tomu sbohem dám.

Prý že tam hrají obtížné rytmy\\
snad sedum osmin, za světla, v přítmí\\
aiolský, dórský, lydický modus\\
a kdo to nezná, ten pije sodu.

On prý je schopný hrát i na rohy\\
někomu hrozně smrděly nohy\\
prý kajtru držím jak prase kosti\\
snad nejsem dobrý pro něho dosti.

Kdo chce s nima zpívat, kdo chce s nima hrát\\
musí jejich písně neomylně znát\\
nadto ještě třeba sluch mít jako rys\\
kdo chce do kapely Canti Amavis.

\columnbreak

Prý se k nim hráči po stovkách derou\\
nakonec praví, tak že mě berou\\
dívá se na mě jak vlk na krůtu\\
a prý mám krátkou na nácvik lhůtu.

Poctivě cvičím, dá se říct denně\\
bolestí kvičím jak malé štěně\\
a když to umím, hodím si mašli,\\
prý už si dávno jiného našli.

Bude s nima zpívat, bude s nima hrát\\
dostane se krásně na jakýkoliv hrad.\\
To se pozná páteř pevná jako tis\\
příště už se vyhnu Canti Amavis.

\end{verse}


\end{multicols}