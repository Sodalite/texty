\documentclass[czech,12pt,landscape]{article}
\usepackage[utf8]{inputenc}
\usepackage[T1]{fontenc}
\usepackage{babel}
\usepackage{xcolor}
\usepackage{libertine}
%\usepackage{bookman}
\usepackage{multicol}
\usepackage{geometry}
 \geometry{a4paper, total={257mm, 170mm}, left=15mm, right=15mm, top=10mm}
\title{Zpěvník}
\author{Lukáš Růžička}

\newcommand\ak[2][l]{%
	\makebox[0pt][#1]{\begin{tabular}[b]{@{}l@{}}\textbf{\color{red}#2}\\\mbox{}\end{tabular}}}

\newcommand{\tak}[1]{\textbf{\color{red}#1}}


\begin{document}
	\begin{titlepage}
		\maketitle
	\end{titlepage}

	\section{Stanu se písničkářem}

\thispagestyle{empty}

\begin{multicols}{2}

\begin{verse}

Kdysi jsem chtěl umět\\
na kytaru hrávat,\\
alespoň tak\\
jako Štepán Rak.\\
Jen každý den cvičit,\\
prsty si drát,\\
potom jsem rychle pochopil,\\
že ujel vlak.

Přesto, že neumím hrát,\\
tak můžu na prknech stát.\\
Vzhůru do světel záře, \\
stanu se písničkářem.

Později chtěl jsem jen\\
krásné písně zpívat,\\
pro všechny vhod,\\
jako Karel Gott.\\
Bohužel vím, že ten \\
tón nechytím,\\
a tak ze sálu marně hledám\\
zadní vchod.

Přesto, že neumím pět,\\
už nechci vrátit se zpět.\\
Musím do světel záře, \\
stanu se písničkářem.

\columnbreak

A potom dokážu\\
bývalým láskám, \\
že pro mě točí se svět.\\
A že už nejsem\\
ten marný klučík,\\
co jsem jím byl tolik let.\\
Třeba to pochopí hned.

Život šel dál a já \\
chtěl jsem zas být básník,\\
pustit se do lyrik\\
jak Viktor Dyk.\\
Za svitu lamp\\ 
však mi kulhal i jamb\\
takže ukážu téhle múze\\
prostředník.

Přesto, že neumím psát,\\
lidem se ukážu rád.\\
Vzhůru do světel záře, \\
stal jsem se písničkářem.


\end{verse}

\end{multicols}
	\section{Kensington Road}

\thispagestyle{empty}

\begin{multicols}{2}


\begin{verse}

Na Kensington Road\\
já poprvé tě viděl stát,\\
na Kensington Road\\
v tobě se zhlíd.\\
A teď nemám klid\\
a nemůžu spát,\\
ať zkouším kamkoliv jít\\
všude vidím tě stát\\
a jako tenkrát\\
zase toužím tě mít\\
na Kensington Road.

Na Kensington Road\\
mně zdálo se, že sním,\\
na Kensington Road\\
přišlo to hned.\\
Bylo mi šestnáct let,\\
tobě možná víc,\\
mně jazyk se plet,\\
když jsem ti zkoušel říct,\\
že hlavu ztrácím\\
a že se snad zblázním\\
na Kensington Road.

Na Kensington Road\\
však oslovil mně kdosi hned,\\
na Kensington Road\\
prý: \uv{Tu holku nech být!\\
Koukej hned po svých si jít\\
a už se nevracej zpět.\\
Ta není pro tebe.\\
No, to ses snad splet.\\
Zaplať a budeš smět\\
s ní třeba do nebe.}\\
Na Kensington Road.

Na Kensington Road\\
já celé týdny jenom dřel,\\
na Kensington Road\\
jsem po parcích spal.\\
A sotva ráno jsem vstal,\\
hned abych zas šel,\\
já každou práci jsem bral,\\
abych ty prachy už měl.\\
A když jsem vztek vybít chtěl,\\
tak jsem se v hospodách rval,\\
na Kensington Road

Na Kensington Road\\
teď kráčím rovnou jako král,\\
na Kensington Road\\
k tobě jdu blíž\\
a ty už možná víš,\\
co bych si přál.\\
Však když se otočíš\\
zas nevím, co ti říct mám,\\
hledím ti do očí\\
a pak zas odcházím sám\\
na Kensington Road

\end{verse}


\end{multicols}
	\section{Milá Sally}

\thispagestyle{empty}

\begin{multicols}{2}
	
	
\begin{verse}		
	Ty ze Sally\\
	píšu ti z dáli\\
	email, který nevím\\
	jak má začínat\\
	snad takhle:
	
	Zrzko ze Sally\\
	včera se mi kolegové smáli\\
	že nemám v lásce štěstí\\
	a prý jsem marný,\\
	že těžko znectím byť jen bábu s klestím.
	
	Krásko ze Sally\\
	asi potřebuju sbalit\\
	a už se neptej\\
	proč se bojím\\
	přijď si pro mě\\
	vždyť sám tu stojím.
	
	V loňském létu\\
	na internetu\\
	jsem viděl tvoji adresu\\
	na kterou mám psát svá snění.\\
	Nač si zoufat\\
	chci teď znovu doufat\\
	že smutný osud můj\\
	konečně změníš.
	
	\columnbreak
	
	Milá ze Sally\\
	jsem tvůj klient stálý\\
	zpívám s drzostí do nebe volající,\\
	že mám tebe.
	
	Lásko ze Sally\\
	co kdybychom spolu -- \\
	posnídali.
	
	Moje Sally.
\end{verse}
	
	
\end{multicols}
	\section{Liška a vrána}

\thispagestyle{empty}

\begin{verse}

Jednou takhle z rána\\
sedí černá vrána\\
a v zobáku svírá\\
velký koláč sýra.

Shora na svět shlíží\\
liška se k ní blíží\\
vránu nahlas zdraví\\
a takto k ní praví:

\uv{Hoj, ty krásný ptáčku,\\
přelibý zpěváčku,\\
zazpívej mi zase\\
jemným hebkým hlasem.}

Přemluvit si dala,\\
vrána zazpívala,\\
dřív než liška vzala sýr,\\
hradní stráž ji sťala,\\
vládce těžko připustí,\\
aby lůza žrala.

\end{verse}
	
	\section {Fronta na vleky}

\thispagestyle{empty}

\begin{multicols}{2}


\begin{verse}
	

Stojím frontu na vleky\\
achich ouha, ta je dlouhá\\
můžu tu být na věky.\\
Necítím však zděšení\\
na Vánoce dlouhý noce\\
znám já jedno řešení.

Mezi nohy zasuň lyži,\\
když se lyže s druhou zkříží,\\
tenhle závod vyhrává,\\
kdo lepší nervy má.\\
Když tě tlačí dav do rohu,\\
nastrč hůlku, nastav nohu,\\
takto spousta minut\\
ušetřit se dá.

Stojím frontu na vleky\\
achich ouha, ta je dlouhá\\
můžu tu být na věky.\\
Necítím však zděšení\\
na Tři krále o krok dále\\
znám já jedno řešení.

Lyže sundej, postup k plotu,\\
použij svoji těžkou botu,\\
tenhle závod vyhrává,\\
kdo lepší nervy má.\\
Pošlap špičky, dup na patky,\\
bude na tě každý krátký,\\
takhle spousta minut\\
ušetřit se dá.

\columnbreak

Stojím frontu na vleky\\
achich ouha, ta je dlouhá\\
můžu tu být na věky.\\
Necítím však zděšení\\
na Hromnice o den více\\
znám já jedno řešení.

Odstrč dítě až do sítě,\\
dítě v síti nebolí tě,\\
tenhle závod vyhrává,\\
kdo lepší nervy má.\\
Nežli vyhrabe se zpátky,\\
zbrzdíš otce, zdržíš matky,\\
takhle spousta minut\\
ušetřit se dá.

Stojím frontu na vleky\\
achich ouha, ta je dlouhá\\
můžu tu být na věky.\\
Já však dobré know-how mám\\
na mou věru, vpřed se deru\\
naděju se, už jsem tam.


\end{verse}


\end{multicols}	
	\section{Limeriky 1}

\begin{multicols}{2}
	


\subsection{Kulinářské pokusy na vsi}

\begin{verse}
Jednou prý by zkusili \\
jak chutnají fusilli.\\
Říká: “Vlasto, tyvole\\
dyť to só jen vrtule.”\\
Na hnůj to pak hodili.
\end{verse}


\subsection{Dívka z Tvrdonic}

\begin{verse}
Včerejšek jsem trávil s dívkou z Tvrdonic,\\
napřed jsem byl tvrdý, potom nic.\\
Ona se mi smála,\\
mou odvahu vzala.\\
Teď už zbývá mi jen, bych se pic’.
\end{verse}

\subsection{Dalibor z Kozojed}

\begin{verse}
Ve věži Dalibor z Kozojed\\
měl hlad, že by švába sněd,\\
z veliké nouze\\
fidlal tak dlouze\\
dokud mu nedali aspoň med.
\end{verse}

\columnbreak

\subsection{Kovboj z Arizony}

\begin{verse}
Jeden kovboj z Arizony \\
jedl jenom burizony. \\
Místní obvoďačka Róza\\
šeptá: “Avitaminóza!\\
Zítra budeš patizony.”
\end{verse}

\subsection{Dámičky z Dallasu}

\begin{verse}
Dvě dámičky z Dallasu \\
dnes si vjely do vlasů.\\
Jedna křičí: “Mé je právo!”\\
Druhá na ni: “Stará krávo!”\\
Svlékly se až do pasu.

\end{verse}



\end{multicols}	
	\section{AD863}

\thispagestyle{empty}

\begin{multicols}{2}


\begin{verse}
	
\ak{dmi}Ty dva při \ak{C}vědomí a \ak{B$^{b}$}zdráv\\
pozval \ak{C}kníže \ak{dmi}Rastislav.\\
\ak{dmi}Bylo \ak{C}dobré \ak{B$^{b}$}povětří\\
á dé \ak{gmi$^{7}$}osum, \ak{A}šest, a \ak{dmi}tři.

\ak{gmi$^{7}$}Jdou, \ak{C}jdou cestou \ak{F}necestou. \ak{dmi} \\
\ak{gmi$^{7}$}Jdou, \ak{C}jdou s Knihou \ak{F}knih. \ak{F\#$^{dim}$}\\
\ak{gmi$^{7}$}Jdou, \ak{C}jdou s nebem \ak{F}nad hlavou. \ak{dmi}\\
\ak{gmi$^{7}$}Jdou, \ak{A}jdou ze směru \ak{dmi}jih.

\vspace{15pt}

Pohanské analfabety\\
přinutili ohnout hřbety\\
jeden moudrý, druhý \uv{in},\\
Metoděj a Konstantin.

Jdou, jdou bratři ze Soluně.\\
Jdou, jdou přes les, přes zahradu.\\
Jdou, jdou při slunci i luně.\\
Jdou, jdou až k Velehradu.

\columnbreak

Dojetím nám zjihly líce\\
při spatření hlaholice\\
od těch dob, a to nás těší,\\
jsme inteligentní Češi,\\
prostě inteligentní Češi,\\
zkrátka inteligentní Češi,\\
velmi inteligentní Češi,\\
vychytralí Češi.
\end{verse}


\end{multicols}
	\section{Moje říčka}

\thispagestyle{empty}

\begin{multicols}{2}

\begin{verse}

Táhne se modrá stužka v údolí\\
je jako šála\\
kdesi se ozve výkřik sokolí\\
bílá je skála\\
seběhnu dolů malou pěšinou\\
k tobě se na břeh posadím\\
a když mě někdy trochu trápí svět\\
můžu ti vyprávět.
 
Vím moje řeko, že mi rozumíš\\
že posloucháš, tím jsem si jistý\\
a všechny moje smutky,\\
strasti a zlé skutky,\\
vezmeš s sebou\\
a já – jsem zas čistý.
 
Důvěru mou, tu nikdy nezradíš\\
má stará známá\\
proudem mi vlasy lehce pohladíš\\
jsi jako máma\\
tvá tichá píseň ta mě konejší\\
a na vlnách mě kolébá\\
a když se s žádnou nemám zrovna rád\\
můžu ti povídat.
 
Vím moje řeko, že mi rozumíš\\
že posloucháš, tím jsem si jistý\\
a všechny moje touhy,\\
bolesti a šmouhy,\\
vezmeš s sebou\\
a já – jsem zas čistý.

\columnbreak

Když se jen skloním nad tvou hladinou\\
vidím čas chvátat\\
už nejsem sám, teď už jsem s rodinou\\
má mě kdo chápat.\\
A přesto chodím k tobě stále dál\\
ačkoliv se mi děti smějí\\
víš, lidé smáli se mi tolikrát\\
chci s tebou vzpomínat.
 
Vím moje řeko, že mi rozumíš\\
že posloucháš, tím jsem si jistý\\
a všechny moje splíny,\\
přehmaty a stíny,\\
vezmeš s sebou\\
a já – jsem zas čistý.
 
Přichází náhle časy podzimní\\
v listí se skrýváš\\
kolik tak týdnů nebo kolik dní\\
do zimy zbývá.\\
A zatímco ty se ztrácíš do moří,\\
my lidé zavíráme víčka.\\
Ještě než dohoří nám svíčka\\
jsi moje říčka.
\end{verse}

\end{multicols}

	\section{Zamilovaný chemik}

\thispagestyle{empty}

\begin{multicols}{2}
	
	\begin{verse}
		
		
		Srdce mám téměř v kalhotách\\
		a zároveň mi v krku buší.\\
		Je to snad touha nebo strach\\
		tak proč mi tedy rudnou uši?\\
		Nevnímám krásu kvetoucích kalin\\
		pokaždé zvedneš mi adrenalin.
		
		Chci se tě stále dotýkat\\
		a všichni už to dávno tuší.\\
		Budeš mi pořád unikat\\
		vždyť stud ti, holka, tolik sluší.\\
		Já chci tě zbavit těsných džín\\
		když ovládá mě dopamin.
		
		Chemie lásky, tělem mi víří\\
		říkají známí, že se mám.\\
		S chemií lásky, jak se s ní smířím\\
		když nejsi k mání a já jsem sám.
		
		Zkouším ti hloupé básně psát\\
		a jsem to já, kdo pak jim věří,\\
		že navždy chci tě milovat,\\
		že budeš se mnou jako v peří.\\
		A nemyslím už na jiné\\
		zaslepen fenylethylaminem.
		
		\columnbreak
		
		Chemie lásky, tělem mi víří\\
		říkají známí, že se mám.\\
		S chemií lásky, jak se s ní smířím\\
		když nejsi k mání a já jsem sám.
		
		Jenže ty mi kazíš plány\\
		prý obelhal mě lásky klam.\\
		Pouštím si horkou vodu do vany\\
		a láhev vína otvírám.\\
		Hrajte mi písně z mollových tónin\\
		už opustil mě i serotónin.
	\end{verse}
	
\end{multicols}
	\section{Vražda tety na minaretu v Lednici}

\thispagestyle{empty}

\begin{multicols}{2}
\begin{verse}

Chtěl jsem včera svoji tetu,\\
protože mi pije krev,\\
shodit dolů z minaretu,\\
leč bránila se jako lev.

Prý věž to je památeční,\\
a má víc než dvěstě let,\\
takto teta ke mně řeční,\\
prý jsem musel zešílet.

Jó, milá teto, proč jsi taková?\\
Nikdo jiný se tím stylem,\\
ke mně nechová.\\
Jó, milá teto, proč jen musíš mít,\\
vždycky jen poslední slovo,\\
k tomu, jak mám žít.

Představím si tři sta schodů,\\
výšku metrů šedesát,\\
co bych musel po obvodu,\\
tlustou tetku vystrkat.

Říkám si, proč kazit dílo\\
stavitele Hardmutha,\\
už se trochu připozdilo,\\
teta ani nedutá.

\columnbreak

Jó, milá teto, proč jsi taková?\\
Nikdo jiný se tím stylem,\\
ke mně nechová.\\
Jó, milá teto, proč jen musíš mít,\\
vždycky jen poslední slovo,\\
k tomu, jak mám žít.

Usměju se, máznu medu,\\
teto, byl to jenom žert,\\
však do vína jí sypu jedu $\ldots{}$\\
dnes v noci ji vezme čert.
\end{verse}
\end{multicols}
	
	\section{Konkurz}

\thispagestyle{empty}

\begin{multicols}{2}


\begin{verse}

Napadlo mě tak nakonec roku,\\
že bych se mohl věnovat rocku.\\
Rád bych si zahrál k poslechu, tanci\\
gotiku, třeba, či renesanci.

Mrknul jsem kolem, kdo to tak hrává,\\
která že grupa bude ta pravá.\\
Nesmí to býti žabaři žádní,\\
nejlepší zdá se ten bigbít hradní.

Šel jsem na konkurz, ztratil den celý,\\
kapelník hlavu (jak) tuřanské zelí.\\
Vytáh’ jsem nástroj, prohrábl struny,\\
a on mi dává pokynů tuny.

Kdo chce s námi zpívat, kdo chce s námi hrát\\
ten se musí dívat na můj prstoklad\\
invenci si nežádám, vše vím nejlíp sám\\
a komu se to nezdá, tomu sbohem dám.

Prý že tam hrají obtížné rytmy\\
snad sedum osmin, za světla, v přítmí\\
aiolský, dórský, lydický modus\\
a kdo to nezná, ten pije sodu.

On prý je schopný hrát i na rohy\\
někomu hrozně smrděly nohy\\
prý kajtru držím jak prase kosti\\
snad nejsem dobrý pro něho dosti.

Kdo chce s nima zpívat, kdo chce s nima hrát\\
musí jejich písně neomylně znát\\
nadto ještě třeba sluch mít jako rys\\
kdo chce do kapely Canti Amavis.

\columnbreak

Prý se k nim hráči po stovkách derou\\
nakonec praví, tak že mě berou\\
dívá se na mě jak vlk na krůtu\\
a prý mám krátkou na nácvik lhůtu.

Poctivě cvičím, dá se říct denně\\
bolestí kvičím jak malé štěně\\
a když to umím, hodím si mašli,\\
prý už si dávno jiného našli.

Bude s nima zpívat, bude s nima hrát\\
dostane se krásně na jakýkoliv hrad.\\
To se pozná páteř pevná jako tis\\
příště už se vyhnu Canti Amavis.

\end{verse}


\end{multicols}
	\section{Kam jdete, ženy?}

\thispagestyle{empty}

\begin{multicols}{2}


\begin{verse}
	
Kam kráčíš, mámo?\\
Teď sám tu stojím.\\
Tvé kroky chodbou\\
doznívají dál.\\
Odcházíš, mámo,\\
trochu se bojím.\\
Tabule s křídou -- \\
a kdo vám o ni stál.

Ale čas jako proud\\
nese nás, musím plout,\\
snad to zas\\
půjde snáz.

Kam kráčíš, lásko,\\
a proč jdeš sama?\\
Sotva tě vítám\\
ty se loučíváš.\\
Odcházíš, lásko,\\
jsi za horama.\\
Poslední sbohem, \\
ty holko marnivá.

Ale čas jako proud\\
nese nás, musím plout,\\
snad to zas\\
půjde snáz.

\columnbreak

Kam kráčíš, dcero?\\
Není to dávno,\\
co jsem tě chránil\\
před tmou v náručí.\\
Odcházíš, dítě,\\
snad cestou správnou.\\
Já nemám nárok,\\
když srdce poručí.

Ale čas jako proud\\
nese nás, musím plout,\\
snad to zas\\
půjde snáz.

Kam jdete, ženy?\\
Kam odcházíte?\\
Jaký měl záměr\\
kdo stvořil tento řád?\\
Kam jdete, ženy,\\
když opouštíte?\\
Bez ohlédnutí\\
nás necháváte stát.

\end{verse}


\end{multicols}	
	\section{Limeriky 2}

\begin{multicols}{2}
	
	
	
	\subsection{Fotograf}
	
	\begin{verse}
	Jeden známý fotograf\\
	chtěl si cvaknout místní splav.\\
	Stala se věc strašná,\\
	spadla mu tam brašna.\\
	Jenom vzdechl: “Tak si plav.”
	\end{verse}
	
	
\subsection{Indiání}
	
\begin{verse}
Čejenové z Koloráda\\
špatně nastavili záda.\\
Bílí kdesi u Sand Creeku\\
bodnuli jim do nich dýku.\\
Prachsprostá to byla zrada.
\end{verse}
	
\subsection{Děda z Olešnice}
	
\begin{verse}
Byl jeden děda z Horní Olešnice\\
huba mu jela jako tarasnice.\\
Když potom na chvilku nic --\\
v zápětí řval o to víc.\\
Kéž by ho radši dali do krabice.
\end{verse}
	
	\columnbreak
	
\subsection{Jan Hus}
	
\begin{verse}
Provokatér, tenhle Hus,\\
dosti drzý na můj vkus.\\
Boháč, chudák, že jim rovno?\\
Na to říkám: “Leda hovno.”\\
Spalte ho jak papyrus.
\end{verse}
	
\subsection{Jazzman v baru}
	
\begin{verse}
Vzal si jazzman do baru\\
svoji novou kytaru.\\
Tak se chlubil pololuby\\
až mu vymlátili zuby.\\
Ručně, stručně, postaru
\end{verse}
	
\end{multicols}
	\section{Milenci s rovnátky}

\thispagestyle{empty}

\begin{multicols}{2}

\begin{verse}
Dva mladí milenci u autobusu\\
chtěli si vyzkoušet francouzskou pusu.\\
Nejprv to bylo jak pohádka,\\
než se jim zasekla rovnátka.

Přece jsou šťastní, i když to bolí,\\
protože taková prý láska bývá.\\
Drží se za ruce uprostřed polí\\
a zatím se nad světem pomalu\\
pomalu\\
pomalu\\
pomalu stmívá.

Chtějí je uvolnit, to jsou mi věci,\\
hlavu si lámají nad tím ježkem v kleci.\\
Zkoušejí doprava, pak dolů,\\
hlavně, že mohou být pospolu.

Přece jsou šťastní, i když to nudí,\\
protože taková prý láska bývá.\\
Drží se za ruce a ruce je studí\\
a nad nimi Měsíc si ospale\\
ospale\\
ospale\\
ospale zívá.

\columnbreak

Nakonec objeví ten postup pravý,\\
ústa se oddálí, celkem jsou zdraví.\\
Pole se kolem nich rozvlní.\\
Kolikpak zbývá jim ještě dní?

Stále jsou šťastní, i přes trochu krve,\\
protože taková prý láska bývá,\\
byť to pak nebude jak tehdy prve\\
kdyžs pro ni na louce nad ránem,\\
nad ránem,\\
nad ránem\\
korále sbíral.
\end{verse}


\end{multicols}

	\section{Výběr z bobulí}

\thispagestyle{empty}

\begin{multicols}{2}
	
\begin{verse}
		
Když je děcko rozmrzelé,\\
když ho zmáhá pláč,\\
nepomáhá pohlazení,\\
pusa ani fáč.\\
Když si už pak enem přeješ,\\
ať už nebulí,\\
tož mu nalej do lahvičky\\
výběr z bobulí.

Když ťa potká první láska,\\
srdce své jí dáš,\\
žiješ jako na obláčku\\
z pěsniček to znáš.\\
Když sa k tebě na procházce\\
pevně přitulí,\\
zachutná ti láska jako\\
výběr z bobulí.

Když to doma zadrhává,\\
když to doma dře,\\
starý čučí po hospodách\\
krev ti zlosťú vře.\\
Když sa potom vrátí domů,\\
smrdí cibulí,\\
spolehlivě přerazí to\\
výběr z bobulí.

Když ťa melú mlýnské kola,\\
když ťa mrzí svět,\\
na veliké, na trápení\\
ťažko zapomnět.\\
A když cítíš, že ťa život\\
kopnul do kulí,\\
tož si na to rychlo nalej\\
výběr z bobulí.

A jak už sa ráno budíš,\\
než sa rozední,\\
pomalu sa přichystáváš\\
k cestě poslední.\\
A když hledíš pod oltářem\\
z černej škatuly,\\
rozlévá sa na tvú počest\\
výběr z bobulí.
	
	
\end{verse}
	
\end{multicols}	
	\section{Přiletěl orel}

\thispagestyle{empty}

\begin{verse}

Přiletěl orel z dalekých plání\\
s východním větrem před rozedněním.\\
Přiletěl orel, tak znenadání,\\
a jeho křídla jsou tichá jak stín.

Přiletěl orel a střemhlav z výšky\\
do mého těla zaťal svůj spár.\\
Přiletěl orel, však s úskokem lišky,\\
na kusy trhá mě nebeský car.

Přiletěl orel a klidně mě vraždí\\
a já se ptám, zda mojí vinou?\\
Loučím se s tebou, možná už navždy\\
a slunce vychází nad krajinou.\\
Loučím se s tebou, možná už navždy\\
a slunce vychází nad Ukrajinou.

\end{verse}

	\section{Já se houpu}

\thispagestyle{empty}

\begin{multicols}{3}


\begin{verse}
	
Dali mi dneska, lásko, vybrat,\\
abych jim prozradil,\\
co nosím ve své hlavě.\\
Že prý to vezme rychlý obrat,\\
když jim to nepovím,\\
pak řekli usměvavě.

Dali mi dneska, lásko, vybrat,\\
abych jim vyklopil,\\
co písní nosím pod čepicí.\\
A za to, že jsem je chtěl zahrát\\
a pravdu ukázat,\\
skončím na šibenici.

A tak se houpu\\
a slunko zlatě září.\\
Já se houpu\\
zvolna nad krajinou.\\
Už se houpu\\
a vítr hladí mě ve tváři.\\
Já se houpu\\
stále nad hlubinou.\\
Aspoň mám pěkný výhled po okolí.\\
Kdo by to řekl, že smrt nebolí?\\
Kdo by to řekl, že smrt nebolí?\\
Kdo by to řekl, že smrt nebolí?

\columnbreak

Dali mi včera, lásko, příkaz,\\
abych jim vyprávěl,\\
co jsem učil děti.\\
A založili na to výkaz\\
a prý to se mnou jde\\
od deseti k pěti.

Prý kdybych se, lásko, držel zpátky,\\
moh’ jsem si užívat\\
a mít se jako prase v žitě.\\
Dali mi hlavu do oprátky\\
a kat mi pošeptal:\\
“Hanba konformitě!”

A tak se houpu\\
a slunko zlatě září.\\
Já se houpu\\
zvolna nad krajinou.\\
Už se houpu\\
a vítr hladí mě ve tváři.\\
Já se houpu\\
stále nad hlubinou.\\
Aspoň mám pěkný výhled po okolí.\\
Kdo by to řekl, že smrt nebolí?\\
Kdo by to řekl, že smrt nebolí?\\
Kdo by to řekl, že smrt nebolí?

Vezmi si kousek toho provazu\\
a nos jej uvázaný na zápěstí.\\
Z ruky mi stáhni snubní prsten,\\
snad přinese ti štěstí.

A tak se houpu\\
a slunko zlatě září.\\
Já se houpu\\
zvolna nad krajinou.\\
Už se houpu\\
a vítr hladí mě ve tváři.\\
Já se houpu\\
stále nad hlubinou.\\
Aspoň mám pěkný výhled po okolí.\\
Kdo by to řekl, že smrt nebolí?\\
Kdo by to řekl, že smrt nebolí?\\
Kdo by to řekl, že smrt nebolí?

\end{verse}


\end{multicols}	
	\section{Drak pod postelí}

\thispagestyle{empty}

\begin{multicols}{3}


\begin{verse}
	
\ak{D}Já mám doma \ak{emi}pod postelí\\
\ak{A7}schovaného \ak{D}draka\\
ten mě vždycky \ak{emi7}navštěvuje\\
\ak{A7}když se večer \ak{D}smráká.\\
\ak{D}Posadí se \ak{G}do ušáku\\
\ak{E}z tlamy se mu \ak{A}kouří\\
\ak{G}krásné kroužky \ak{D}vyfu\ak{hmi}kuje\\
\ak{emi7}líně oči \ak{A7} mhou\ak{D}ří.

Povídám mu o trápeních\\
o nesplněných touhách.\\
On mi na to odpovídá,\\
že se nesmím rouhat.\\
Říká, že se přání plní,\\
kde se o ně žádá.\\
Pak se opře, zkříží nohy,\\
jak to kreslí Lada.

\columnbreak

\ak{G}Tenhleten \ak{D}drak\\
\ak{A}ten mi vždycky \ak{D}porozumí\\
\ak{G}když se ten \ak{D}svět\\
tak \ak{emi}těžko \ak{E7}dobý\ak{A}vá.\\
\ak{G}Jó, ten můj \ak{D}drak\\
\ak{F\#}rozveselit \ak{hmi}vážně umí\\
\ak{G}létám s ním \ak{D}tak jak pták\\
\ak{A}až do \ak{D}oblak.

Stýskám si, že moje žena\\
nevěří mým písním.\\
A já se je stydím zahrát\\
třeba i jen místním.\\
Usměje se spiklenecky, jak\\
na stranické schůzi:\\
\uv{Klidně nechej ženu plkat,\\
hlavní jsou tvé múzy.}

Tenhleten drak\\
ten mi vždycky porozumí\\
když se ten svět\\
tak těžko dobývá.\\
Jó, ten můj drak\\
rozveselit vážně umí\\
létám s ním tak jak pták\\
až do oblak.

Chvíli ho jen pozoruju\\
unaveným zrakem.\\
Ptá se, jestli nelituju,\\
že tu sedím s drakem.\\
Pokouším se odpovědět\\
slova se mi pletou.\\
Už jen cítím, jak mě bere\\
létat nad planetou.

Tenhleten drak\\
ten mi vždycky porozumí\\
když se ten svět\\
tak těžko dobývá.\\
Jó, ten můj drak\\
rozveselit vážně umí\\
létám s ním tak jak pták\\
až do oblak.\\
Jen já vím, že můj drak\\
voní po švestkách.
\end{verse}


\end{multicols}
	\section{La Belle Dame Sans Merci}

\begin{multicols}{2}
	\begin{verse}
		Co trápí tě dnes, rytíři, \\
		že ztrácíš svůj čas v blouzněních,  \\
		a rákos sklání stébla svá  \\
		i pták už ztich.
		
		Co trápí tě dnes, rytíři,  \\
		že tvář máš bledou neštěstím,\\  
		vždyť úroda je pod střechou,  \\
		konec dlouhým žním.
		
		Lesklé slzy na řasách,  \\
		jak bílé lístky kopretin. \\ 
		I růže v tváři mění se  \\
		na pouhý stín.
		
		Já potkal krásku na lukách\\  
		tak nádhernou jak dítě víl, \\ 
		a oči měla hluboké  \\
		a já z nich pil.
		
		Já držel jsem ji v náručí\\  
		a celý den jsem s ní jen stál,  \\
		pak zpívala mi píseň svou  \\
		a já ji znal.
		
		Pak náramek jsem uvázal  \\
		a do vlasů jí věnec vplet. \\ 
		V očích měla lásky zář  \\
		a v ústech med.
		
		Nabídla mi sladký pel  \\
		a sváděla mě k něžným hrám  \\
		a cizí řečí povídá:  \\
		"Já tě ráda mám."
		
		Pak najednou ji přemoh' pláč,  \\
		když mě vzala na svůj hrad.  \\
		Já směl ji líbat na víčka,  \\
		a kolébat.
		
		Já s ní jsem ležel v peřinách \\ 
		a dlouho spal, ach bohužel,  \\
		mně zdál se sen, už poslední,  \\
		co kdy jsem měl.
		
		V něm bledých králů, princů voj  \\
		armáda snad stohlavá  \\
		volá: "La belle Dame sans merci  \\
		tě spoutává."
		
		V tvářích měli všichni strach  \\
		v očích plál jim divný klam  \\
		a potom jsem se probudil  \\
		a byl jsem sám.
		
		Teď stesk mě trápí, po ní jen,  \\
		a ztrácím svůj čas v blouzněních, \\ 
		i rákos sklání stébla svá  \\
		i pták už ztich.
		
	\end{verse}
\end{multicols}
	\section{Limeriky 3}

\begin{multicols}{2}
	
\subsection{Lidojed}
	
\begin{verse}
Dojetím se třeseš hned -- \\
samou láskou by tě sněd.\\
Jenom ti to klidu nedá\\
když tu větu, dvakrát běda,\\
když ji řekne lidojed.
\end{verse}
	
	
\subsection{Lupič z Arkansasu}
	
\begin{verse}
Mladý lupič z Arkansasu \\
šel tak v noci vybrat kasu.\\
Nevšimnul si kamer v koutech\\
už ho vede šerif v poutech\\
teď už bude hrát jen basu.
\end{verse}
	
\subsection{Kráva a psychosomatika}
	
\begin{verse}
Povídala kráva krávě,\\
je jí zle po čerstvé trávě.\\
Energii rychle ztrácí\\
prý až z knihy se to vrací.\\
Říká druhá: “Máš to v hlavě.”
\end{verse}
	
\columnbreak
	
\subsection{Opilec}
	
\begin{verse}
Vypil pivo, víno, vodku,\\
nevšiml si potom schodku,\\
za okamžik pouhý,\\
široký a dlouhý --\\
hodili ho k plotku.
\end{verse}
	
\subsection{Básník limeriků}
	
\begin{verse}
Chlubil se mi jeden chlápek v triku,\\
že vymyslí tisíc limeriků.\\
A pak mi slíbil v převeliké pýše,\\
jestliže jich tisíc nenapíše,\\
tak udělá za trest deset kliků.
\end{verse}
	
\end{multicols}
	\section{Vezmi mě, táto}

\thispagestyle{empty}

\begin{multicols}{2}
\begin{verse}

Vezmi mě táto do Londýna\\
život tam nikdy neusíná\\
projdeme se spolu aspoň chvíli\\
po Oxford Street a po Picadilly

Sherlockem můžem být na Baker Street\\
koupíme selfie tyče u Selfriče\\
dáme si polárku třeba v Hyde Parku\\
navštivme koutek mistrů ve Westminstru\\
naberem správný směr na Trafalgar Square\\
pak do Toweru jsme zvaný na havrany\\
hrozně mě srdce svírá na hře od Shakespeara\\
není to žádný trik překročit poledník.

Vezmi mě táto jen do Vídně\\
Sachrův dort posnídáme šestkrát týdně\\
a potom sedmý den si zajdem spolu\\
do Prátru svézt se na ruském kolu.

Zdáme se malí tam u katedrály\\
flašinety nám hrají při Dunaji\\
a centrem po staru jedeme v kočáru\\
obrazů plné síně v Albertině.\\
A Mozart který láká do opery\\
dějiny se nám smějí u muzeí\\
ochutnáváme svět v Café Gloriette\\
a kdosi vdává dceru v Belvedéru.

Vezmi mě táto do Bratislavy\\
říkala máma, že se stavíš\\
za týden, za dva, do měsíce\\
jen co tě propustí z nemocnice.

\columnbreak

Exkurze nám začíná u Martina\\
hudbou to žije u filharmonie\\
svět u nohou nám leží pod Michalskou věží\\
pak vem mě na hrad do barokních zahrad\\
Čas se pak vleče líně na Slavíně\\
v talíři buď můj host, pojď na Nový most\\
a věž gotikou sálá v Sadu Janka Kráľa\\
to nakonec se hodí na Děvín jet lodí.

Vezmi mě třeba jen po Brně\\
půjdeme pomalu a opatrně\\
podej mi ruku, já dám ti svoji,\\
chytím ti kuličku na orloji.

V kašně je Voda celá od Skácela\\
pod Jakubem nás kosti děsí dosti\\
uteče dlouhá chvíle tam pod krokodýlem\\
dáme si do zobáku na Zelňáku.\\
Kometu čekám marně na hvězdárně\\
a mamut z kokosu je v Antroposu\\
je fajn, že vzpomněl sis na Ignis Brunensis\\
pojď rychle už se šeří nad Veveřím.

Vezmi mě táto do náruče\\
poslechni jak mi srdce tluče\\
přijď za mnou prosím jak jsi slíbil\\
vždyť dobře víš, jak moc mi chybíš.

\end{verse}
\end{multicols}

	\section{Jedu domů}

\thispagestyle{empty}

\begin{multicols}{2}

\begin{verse}

Já jedu domů,\\
ale moje srdce svírá velká tíž,\\
která roste víc, čím domovu jsem blíž.\\
Dívám se z okna a náhle dobře vím,\\
že téhle váhy už se nezbavím.\\
Já jedu domů.

Já jedu domů,\\
vlak v stráni víří napadaný sníh,\\
a v uších slyším zas tvůj uštěpačný smích,\\
když mi říkáš, že jsem jak nezvaný host,\\
a že mě máš … už dost.

Proč nelze vše vzít zpět?\\
Proč nelze vrátit čin?\\
Kdo bude rozumět\\
té sbírce našich vin,\\
která se načítá\\
do knihy stížností?\\
Kdo mě kdy pochopí?\\
Kdo mi kdy odpustí?

Já jedu domů,\\
a tebe nechal jsem tam samotnou,\\
tvůj poslední dech zvolna mizí tmou.\\
Ze všeho nejvíc bych teď ještě chtěl,\\
abych ti byl lépe rozuměl.\\
Já jedu domů.

\columnbreak

Já jedu domů\\
a v duši nesu ten nejstarší hřích.\\
Na čele znak následníků Kainových\\
a ze svých rukou mám najednou strach,\\
neboť mě usvědčují, že přestoupil jsem práh.

Proč nelze vše vzít zpět?\\
Proč nelze vrátit čin?\\
Kdo bude rozumět\\
té sbírce našich vin,\\
která se načítá\\
do knihy stížností?\\
Kdo mě kdy pochopí?\\
Kdo mi kdy odpustí?

Já chtěl bych domů,\\
ale vlastně nevím, co bych jim měl říct?\\
Že jsem všechno zničil a nezůstalo nic.\\
A tak skočím do té bílé závěje,\\
kde se v nekonečnu protnou koleje.\\
Dojel jsem domů.

\end{verse}

\end{multicols}			
	\section{Vlasy Bereniky}

\thispagestyle{empty}

\begin{multicols}{2}
\begin{verse}

Dívám se do noci hvězdnaté,\\
kolem projel velký vůz,\\
v patách jdou mu psové honící,\\
zkus jim, vozko, ujet, zkus.\\
Dávný zápas hlavou se mi přehrává\\
jak se ztrácíš v dálce\\
a jen ten příběh zůstává.

Vzpomínáš, Bereniko\\
jak jsem ti hladil vlasy\\
teď stále hledám znova\\
co tu dávno není.\\
Vzpomínám, Bereniko\\
a čekám, kdo mě spasí\\
zůstala jenom slova\\
rozloučení.

Stál jsem s tebou skoro bez hnutí,\\
když střelec šípem proťal vzduch\\
teď orel miluje se s labutí\\
a uzavřel se dávný kruh.\\
Velkou láskou planu jako meteor\\
zatím v trávě blízko\\
ukrývá se lstivý tvor.

Vzpomínáš, Bereniko\\
jak jsem ti hladil vlasy\\
teď stále hledám znova\\
co tu dávno není.\\
Vzpomínám, Bereniko\\
a čekám, kdo mě spasí\\
zůstala jenom slova\\
rozloučení.

Dělá se pokaždé nestranným,\\
když ti v hříchu radí had.\\
Slovům uvěřil jsem šeptaným\\
liška, že je zvyklá lhát.\\
Trojúhelník a nad přeponou vládne drak\\
v hlavách štír\\
a v mlhovinách slábne zrak.

Vzpomínáš, Bereniko\\
jak jsem ti hladil vlasy\\
teď stále hledám znova\\
co tu dávno není.\\
Vzpomínám, Bereniko\\
a čekám, kdo mě spasí\\
a čekám marně slova\\
odpuštění.

Dívám se do noci hvězdnaté,\\
kolem projel velký vůz,\\
v patách jdou mu psové honící,\\
zkus jim, vozko, ujet, zkus.\\
Dávný zápas hlavou se mi přehrává\\
jak se ztrácíš v dálce\\
a jen ten příběh zůstává.

\end{verse}
\end{multicols}

	\section{Žába na prameni}

\thispagestyle{empty}

\begin{multicols}{2}
	
\begin{verse}
		
Proč není v studni živá voda,\\
zeptej se děda Vševěda,\\
proč tam kde dříve rostly růže,\\
dnes roste jenom lebeda.\\
proč už v tom městě zlatá jabloň\\
své plody dávno nedává?\\
a proč lidem srdce zkameněla,\\
když byla dříve laskavá?

Těžko se někdy něco změní,\\
dokud bude sedět žába na prameni.

Proč vždycky rychle zapomene\\
kat, že sám býval zlodějem\\
a když pán slouhu bije bičem\\
ni šašek už se nesměje\\
proč jen se počestněji chová\\
kdejaká děvka v negližé\\
a proč se vždycky najde zrádce,\\
co katům ruce olíže?

Těžko se někdy něco změní,\\
dokud bude sedět žába na prameni.

\columnbreak

Proč někteří si stále myslí\\
že svět se pro ně otáčí\\
proč peníze si doma syslí\\
a ještě jim to nestačí\\
říkáš si, že to bude lepší,\\
starý převozník umírá\\
a pak se s překvapením díváš\\
jak horší vesla přebírá.

Těžko se někdy něco změní,\\
dokud bude sedět žába na prameni.		
	
\end{verse}
	
\end{multicols}	
	\section{Poštovní}

\thispagestyle{empty}

\begin{multicols}{2}

\begin{verse}

Mezi námi dvěma i pohledy už váznou,\\
včera jsem o tobě dlouho snil,\\
ráno vstanu a schránku zas prázdnou mám,\\
i když pošťák už tu dávno byl.

Když jsem tě prvně viděl, to byl pocit, jak když letím,\\
šel jsem hned domů dopis psát.\\
A potom zpátky. Máte jen do pěti.\\
Ještě ho musím odeslat.

Telegramy, složenky hlava z toho šílí,\\
dopisy a účtenky chci být s tebou chvíli,\\
a tenhleten koresponďák připočítej k tomu,\\
doručenou zásilku vezmu ti hned domů.

Stihnul jsem to přesně za pět minut dvanáct,\\
na okénko zaťukal.\\
Nad levou kapsou máš, že jmenuješ se Jana,\\
tak jsem ti ten dopis dal.

A ty jsi řekla: „Je v tom malá potíž. \\
Ten dopis nemá adresu.“\\
A já na to: „To je schválně, totiž,\\
domů ti ho sám zanesu.“

Telegramy, složenky hlava z toho šílí,\\
dopisy a účtenky chci být s tebou chvíli,\\
a tenhleten koresponďák připočítej k tomu,\\
doručenou zásilku vezmu ti hned domů.

\columnbreak

Pak jsem tě domů prvně doprovázel,\\
a potom skoro každý den.\\
Večer jsme spolu pluli v mléčné dráze,\\
tak to šlo týden za týdnem.

Jednou jsem tě zase před poštou čekal,\\
bylo to v pátek, co já vím,\\
čas rychle plynul, tak jako plyne řeka,\\
a ty jsi zatím byla s ním.

Rozchody a zklamání, tak to přece bývá,\\
stojím pořád před poštou a pomalu se stmívá.\\
Lidi chodí na noční a oči města svítí,\\
osud chystá sítě a všechny nás v nich chytí.

Mezi námi dvěma teď i pohledy váznou,\\
včera jsem večer dlouho pil,\\
ráno vstanu a v hlavě vzduchoprázdno\\
mám $\ldots{}$ 
\end{verse}

\end{multicols}

	\section{U řeky Moravy}

\thispagestyle{empty}

\begin{verse}

Na starém hradišti\\
u řeky Moravy\\
viděl jsem zvečera\\
přilétat husy.\\
Už jako malý kluk\\
vzal jsem si do hlavy;\\
proč vždycky na podzim\\
odlétnou musí?

Teď už jsem dospělý\\
a měl bych leccos znát.\\
Třeba to o husách,\\
že mají to v genech.\\
A jak jsme na tom my?\\
Sbohem je napořád!\\
Naše dva rybníčky\\
jsou opuštěné.
\end{verse}


	\section{Černobíle}

\begin{verse}
Jsme vedle sebe, téměř se dotýkáme,\\
a za ta léta už se tak nějak známe,\\
za jeden provaz, my stále táhnem spolu,\\
to podle nálady, buď zvesela či v mollu.

V období dešťů si slzy otíráme,\\
vždycky nás mrzí, když jen se pohádáme,\\
spolu se smějem a dál se máme rádi,\\
nejhezčí chvíle jsou, když nám to dobře ladí.

Ptáme se znova a zase, milá Jano,\\
kdy položíš si své ruce na piáno,\\
černá a bílá, my stále máme víru,\\
ona a já — klávesy na klavíru.
\end{verse}


	\section{Zambezi}

\begin{multicols}{2}
\begin{verse}
Zkoušela sis na mě svoje triky,\\
já jsem jel radši do Afriky\\
a ty doma hniješ si na mezi\\
a já se plavím po Zambezi.

K vodopádu je to jednu míli.\\
Jak jsou milí, vážení,\\
ti krokodýli zelení.\\
K vodopádu je to jen půl míle.\\
Zaberte, hej rup, hoši,\\
ať nás nechytí ti rozzuření hroši.

Když jsem doma sušil černé bezy,\\
nevědělas, co v tom vězí,\\
a já teď s těma všema penězi\\
mířím rovnou na Zambezi.

Na brigádě jsem sušil černý bez,\\
na pravé straně je les,\\
nalevo holky nahoře bez.\\
Plavu si v řece oddaně,\\
je to lepší než doma ve vaně.\\
Hlavně, že tu nejsou piraně.

\columnbreak

Když pojedete do Afriky,\\
nezapomeňte na papriky.\\
Já mám totiž svoji tezi,\\
že nejlíp chutná na Zambezi.

I když se sítí přikryji,\\
komáři krev mi vypijí.\\
Je to stejné jako doma na Dyji.\\
K vodopádu je to jen čtvrt míle,\\
blíží se velmi nebezpečná chvíle,\\
a to tak rychle, až to ani není milé.

Vracíme se domu vítězně,\\
všecno se nám líbí bezmezně.\\
Bezmězně je bez všech mezí,\\
tak krásně je na Zambezi.

\end{verse}
\end{multicols}


\end{document}