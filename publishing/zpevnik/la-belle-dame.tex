\section{La Belle Dame Sans Merci}

\begin{multicols}{2}
	\begin{verse}
		Co trápí tě dnes, rytíři, \\
		že ztrácíš svůj čas v blouzněních,  \\
		a rákos sklání stébla svá  \\
		i pták už ztich.
		
		Co trápí tě dnes, rytíři,  \\
		že tvář máš bledou neštěstím,\\  
		vždyť úroda je pod střechou,  \\
		konec dlouhým žním.
		
		Lesklé slzy na řasách,  \\
		jak bílé lístky kopretin. \\ 
		I růže v tváři mění se  \\
		na pouhý stín.
		
		Já potkal krásku na lukách\\  
		tak nádhernou jak dítě víl, \\ 
		a oči měla hluboké  \\
		a já z nich pil.
		
		Já držel jsem ji v náručí\\  
		a celý den jsem s ní jen stál,  \\
		pak zpívala mi píseň svou  \\
		a já ji znal.
		
		Pak náramek jsem uvázal  \\
		a do vlasů jí věnec vplet. \\ 
		V očích měla lásky zář  \\
		a v ústech med.
		
		Nabídla mi sladký pel  \\
		a sváděla mě k něžným hrám  \\
		a cizí řečí povídá:  \\
		"Já tě ráda mám."
		
		Pak najednou ji přemoh' pláč,  \\
		když mě vzala na svůj hrad.  \\
		Já směl ji líbat na víčka,  \\
		a kolébat.
		
		Já s ní jsem ležel v peřinách \\ 
		a dlouho spal, ach bohužel,  \\
		mně zdál se sen, už poslední,  \\
		co kdy jsem měl.
		
		V něm bledých králů, princů voj  \\
		armáda snad stohlavá  \\
		volá: "La belle Dame sans merci  \\
		tě spoutává."
		
		V tvářích měli všichni strach  \\
		v očích plál jim divný klam  \\
		a potom jsem se probudil  \\
		a byl jsem sám.
		
		Teď stesk mě trápí, po ní jen,  \\
		a ztrácím svůj čas v blouzněních, \\ 
		i rákos sklání stébla svá  \\
		i pták už ztich.
		
	\end{verse}
\end{multicols}