\section*{Právo k zabíjení}

\textit{License to Kill (Bob Dylan)}

\begin{verse}
 Člověk věří, že když řídí svět,\\ 
 co bude chtít v něm udělá.\\
 A když se věci rychle nemění, \\
 změní je on.\\
 Člověk si vymyslel vlastní krach\\
 už na Měsíci, když překročil práh.\\
 Znám tu ženu,\\
 od nás z domu,\\
 jenom tak sedí,\\
 skoro až do setmění,\\
 a ptá se kdo \\
 mu může vzít \\
 právo k zabíjení?
 
 Pak ho vezmou, všechno naučí \\
 a vypustí ho, aby žil. \\
 A z té cesty, kterou mu ukážou,  \\
 on onemocní. \\
 Nádherný pohřeb je poslední cíl, \\
 srdce mu vezmou jako náhradní díl. \\
 Znám tu ženu, \\
 od nás z domu, \\
 jenom tak sedí, \\
 navzdory mžení, \\
 a ptá se kdo \\
 mu může vzít \\
 právo k zabíjení?
 
 Teď je chtivý do ničení, \\
 má strach, je zmatený. \\
 A jeho mozek, ten mu dovedně \\
 ovládají. \\
 Věří svým očím, že pravdivé jsou, \\
 a jeho oči, ty mu do očí lžou. \\
 Já znám tu ženu, \\
 od nás z domu, \\
 co jen tak sedí, \\
 zima jí není, \\
 a ptá se kdo \\
 mu může vzít \\
 právo k zabíjení?
 
 Můžeš být duchovní, podvodník \\
 pohádkář, jiný lhář, \\
 myslet, že světu poručíš. \\
 Můžeš svou roli skvěle hrát, \\
 však někdo jiný se má smát, \\
 než se z chyb svých poučíš.
 
 Teď se modlí na oltáři  \\
 té vody stojaté. \\
 Když svůj odraz dole uvidí, \\
 je naplněný. \\
 Člověk ten neumí férově hrát, \\
 chce všechno mít a nechce nic dát. \\
 Znám tu ženu, \\
 od nás z domu, \\
 co jen tak sedí, \\
 skoro do kuropění, \\
 a ptá se kdo  \\
 mu může vzít \\
 právo k zabíjení?
\end{verse}
